
%%%%%%%%%%%%%%%%%%%%
%                                                                  % 
%      Orion GSAOI                                    %
%                                                                 %
%                                                                  % 
%%%%%%%%%%%%%%%%%%%%

\documentclass[12pt,preprint]{aastex}
%\documentclass{$HOME/A_WORK_DIRECTORIES/Latex/emulateapj}
%$
%% manuscript produces a one-column, single-spaced document:

\usepackage{rotating}

\citestyle{aa}

\bibliographystyle{$HOME/A_WORK_DIRECTORIES/Latex/apj_w_etal}
%$

%\documentclass[preprint2]{aastex}
%% preprint2 produces a double-column, single-spaced document:

%\documentclass[manuscript]{aastex}
%% preprint2 produces a one-column, double-spaced document:

\newcommand{\vdag}{(v)^\dagger}
\newcommand{\myemail}{skywalker@galaxy.far.far.away}
\newcommand{\lsim}{{_{<}\atop^{\sim}}}
\newcommand{\gsim}{{_{>}\atop^{\sim}}}
\newcommand{\etal}{{et al.\/}}
\newcommand{\ie}{{\em ie.\/}}
\newcommand{\cmq}{cm{$^{-3}$}}
\newcommand{\per}{$^{\rm{-1}}$}
\newcommand{\tc}{{$\theta^1$~Orionis~C}}
\newcommand{\msol}{M{$_{\odot}$}}
\newcommand{\lsol}{L{$_{\odot}$}}
\newcommand{\kms}{km~s{$^{-1}$}}
\newcommand{\hii}{H~{\sc ii}}
\newcommand{\Hii}{H~{\sc ii}}
\newcommand{\Ha}{\mbox{H$\alpha$}}
\newcommand{\sii}{S~{\sc ii}}
\newcommand{\Feii}{[Fe~{\sc ii}]}
\newcommand{\oi}{[O~{\sc i}]}
\newcommand{\nii}{[N~{\sc ii}]}
\newcommand{\oiii}{[O~{\sc iii}]}
\newcommand{\mgii}{[Mg~{\sc ii}]}
\newcommand{\tco}{{$^{13}$CO}}
\newcommand{\CO}{{$^{12}$CO}}
\newcommand{\Tco}{{$^{12}$CO}}
\newcommand{\co}{C{$^{18}$O}}
\newcommand{\Lsol}{L$_{\odot}$}
\newcommand{\Msol}{M$_{\odot}$}
\newcommand{\um}{$\mu$m}
%\newcommand{\C18o}{C$^{18}$O($1\rightarrow 0$)}
\newcommand{\Check}{{\bf ???}}
\newcommand{\mum}{\ensuremath{\mu \mathrm{m}}}
\newcommand{\flux}{flux density}
\newcommand{\solar}{\ensuremath{\odot}}

\newcommand{\msun}{\ensuremath{M_{\odot}}}	 	%  Msun
\newcommand{\lsun}{\ensuremath{L_{\odot}}}			%  Lsun
\newcommand{\lbol}{\ensuremath{L_{\mathrm{bol}}}}	%  Lbol
\newcommand{\ks}{K\ensuremath{_{\mathrm{s}}}}		%  Ks
\newcommand{\hh}{\ensuremath{\textrm{H}_{2}}}			%  H2
\newcommand{\water}{H$_{2}$O}		%  H2O
\newcommand{\feii}{\ion{Fe}{2}}		%  FeII
\newcommand{\sqcm}{cm$^{2}$}		%  cm^2
\newcommand{\percc}{\ensuremath{\textrm{cm}^{-3}}}
\newcommand{\persc}{\ensuremath{\textrm{cm}^{-2}}}
\newcommand{\persr}{\ensuremath{\textrm{sr}^{-1}}}
\newcommand{\peryr}{\ensuremath{\textrm{yr}^{-1}}}
\newcommand{\persec}{\ensuremath{\textrm{s}^{-1}}}
\newcommand{\htwo}{\ensuremath{\textrm{H}_2}}    % H2
\newcommand{\Htwo}{\ensuremath{\textrm{H}_2}}    % H2
\newcommand{\HtwoO}{\ensuremath{\textrm{H}_2\textrm{O}}}   
\newcommand{\htwoo}{\ensuremath{\textrm{H}_2\textrm{O}}}   
\newcommand{\ha}{\ensuremath{\textrm{H}\alpha}}
\newcommand{\hb}{\ensuremath{\textrm{H}\beta}}
\newcommand{\twelveco}{\ensuremath{^{12}\textrm{CO}}}
\newcommand{\thirteenco}{\ensuremath{^{13}\textrm{CO}}}
\newcommand{\ceighteeno}{\ensuremath{\textrm{C}^{18}\textrm{O}}}

	% End definitions



\slugcomment{DRAFT: \today}
\shorttitle{Orion GEMS GSAOI}
\shortauthors{Bally et al.}


\begin{document}

\title{The Orion Fingers:   Near-IR Adaptive Optics Imaging}

\author{
               John Bally\altaffilmark{1},               
               Adam Ginsburg\altaffilmark{2},
	    }
	    
\affil{{$^1$}{\it{
      Department of Astrophysical and Planetary Sciences,\\
      University of Colorado, UCB 389, \\
      Boulder, CO 80309}      
      \email{John.Bally@colorado.edu}}  }
\affil{{$^2$}{\it{
      Department of Astrophysical and Planetary Sciences, \\
      University of Colorado, UCB 389, \\
      Boulder, CO 80309}      
      \email{Adam.Ginsburg@colorado.edu}}  }


\keywords{ISM: - molecular clouds --
ISM: - shocks, outflows
ISM: individual -- Orion Nebula, OMC1
stars: formation -- }

\begin{abstract}
    We present new narrow-band \hh ,   \feii, and broad-band K$_s$ images of the 
    Orion  OMC1 outflow  obtained with the Gemini South multi-conjugate 
    adaptive optics (AO) system and near-infrared imager GSAOI.   These  images 
    reach the diffraction limit of the 8-meter telescope at 2 \mum\ of about 0.05\arcsec  .
    Comparison with previous AO-assisted observations of sub-fields with the Gemini
    North telescope between 2007 to 2009 and ground-based observations going back
    to 1999 enable measurements of proper motions of \hh\ and \Feii\ features.   
    Several sub-arcsecond \hh\ features and many \Feii\
    `fingertips' on the projected outskirts of the flow show proper motions 300 \kms\ or more.  
    The propagation of compact  knots (`bullets') as far as 140\arcsec\ from their ejection 
    sites through the dense OMC1 core sets a lower bound on their densities.  The 
    density, mass, and energy constraints are  consistent with the  disruption  of dense 
    circumstellar disks from within a few AU of massive stars during a final multi-body
    dynamic encounter that ejected  the BN object and radio source I from the OMC1 
    about 500 years ago.  
 \end{abstract}



\section{Introduction} 

The BN/KL region behind the Orion Nebula, located at a distance of about
414 pc \citep{Menten2007} contains a spectacular, 
wide opening-angle, arcminute-scale  outflow emerging from the
OMC1 cloud core.  The flow is traced by the millimeter and
sub-millimeter emission lines of molecules 
such as CO, CS, SO, SO$_2$, and HCN  that exhibit  broad  
($>$ 100 km s$^{-1}$)  emission line wings 
\citep{KwanScoville1976, WisemanHo1996, FuruyaShinnaga2009},  
high-velocity OH, H$_2$O,  and SiO maser  emission \citep{Genzel1981,Greenhill1998}, 
and bright shock-excited `fingers' of  \hh\   and `fingertips' of 1.64 \um\  \Feii\   emission
\citep{AllenBurton93, Bally2011}.    The OMC1 outflow with a southeast 
(red-shifted) - northwest (blue-shifted) axis
contains  at least 8 M$_{\odot}$ of  accelerated gas with a median velocity of about 
20 km s$^{-1}$.     Interferometric CO images, H$_2$O and the 18 km~s$^{-1}$ 
SiO masers,  and dense-gas tracers such as  thermal SiO reveal a 
smaller (8\arcsec\  long) and younger ($\sim$ 200 
year old) outflow along a  northeast-southwest axis emerging from radio source I  
orthogonal to the arc-minute-scale  CO outflow  \citep{BeutherNissen2008,Plambeck2009}.    
The  momentum and kinetic   energy content of these flows is at least  
$160$ M$_{\odot}$ km s$^{-1}$ and  $4 \times 10^{46}$  ergs  \citep{Snell1984} to  
$4 \times 10^{47}$ ergs  \citep{KwanScoville1976}.     \citet{Zapata2009}  presented a
CO J = 2--1 interferometric study and found a dynamic age of  about $500$ 
years for the larger OMC1 outflow.   They noted its impulsive nature, that its  structure is 
different from accretion-disk  powered  flows,  and that it  originated several arc-seconds 
north of the  OMC1 hot-core.   

The OMC1 outflow  contains  a complex of shocks which may indicated an explosive
origins.     High precision astrometric measurements have shown that
the three brightest radio-emitting stars in OMC1, sources BN, I, and possibly
source n, have  proper motions  of 25, 15, and 26 km s$^{-1}$ 
away from a region less than 500 AU in diameter from which they were ejected 
about  500 years ago (Rodriguez et al. 2005; Gomez et al. 2005; 2008).   
\citet{Bally2011} and \citet{Goddi2011} proposed that the explosion was triggered 
by the dynamical rearrangement of a nonhierarchical system of massive stars in 
OMC1 which resulted in the formation of a compact, massive binary (most likely 
source I) and the ejection of both BN,   source I, and possibly source n.  
Proper motion measurements show that the OMC1 finger have a dynamic age
of about 500 years \citep{Bally2011}, consistent with this scenario.   

On the other hand, \citet{Tan2004} proposed that the decay 
occurred about 4,000 years ago in the Trapezium cluster located in the center of
the Orion Nebula and that the OMC1 explosion was
triggered by the close passage of the BN object through the OMC1 core.   Although 
this scenario requires a highly unlikely close encounter of  BN with source I, 
\citet{Tan2013} show that the the parameters of the Trapezium and BN are compatible with 
this scenario.
 
Though rare, the explosive outflow morphology of the  OMC1 outflow is not unique;
other likely examples \citep{Bally2011}.   However, Orion BN/KL is the nearest and least 
obscured,and thus most accessible for high-resolution studies.  
In this {\it Letter}, we present 0.05 arc second resolution images of the 
entire OMC1 outflow in the 1.64 \um\ \Feii\ and 2.12 \um\ \hh\  narrowband filters 
and a broad-band K$_s$ obtained with adaptive optics (AO) on the Gemini South  8 
meter telescope.    This data is combined with  older AO-assisted observations
obtained on Gemini North and natural seeing-limited images acquired with a variety
of other telescope to measure new proper  motions. 


\section{Observations}

\subsection{Gemini South GEMS}

Three 80\arcsec fields were observed in OMC1 between 30 December 2012 and
xx February 2013 using the Gemini South adaptive optics imager GSAOI.

Describe observations (dates, FOVs, etc. exposure times per pixel 
in a Table).    Cite GEMS/GSAOI papers.  Describe reduction steps and
difficulties.

Data were processed with the Gemini pipeline.  Additional astrometric
corrections were required.  We aligned each individual exposure with the
\citet{Muench2002a} catalog sources to acquire a world coordinate system with
RMS pointing error $\sim0.1$ \arcsec.  These images were then stacked.
However, this level of accuracy was inadequate for distortion correction within
the images, so we extracted our own catalog from the co-added, aligned images,
and re-aligned each frame to the new, more precise, and deeper catalog.

\subsection{Gemini North Altair}
We report Gemini North Altair AO imaging of several 40\arcsec\ fields obtained 
between 2007 and 2009. 

Describe Altair Observations.

\section{Results}


For this analysis, images of the OMC1 outflow obtained with the 
Subaru 8 meter telescope on MJD = 51484 in 1999 \citep{Kaifu2000},
the Apache Point Observatory 3.5 meter on MJD = 53331 in 2004  \citep{Bally2011}, 
Gemini North on MJD 54165, MJD 54754, and MJD 55138 between 2007 and 2009, 
and Gemini South on MJD 56323 are used.   These observations span an interval of
4,839 days.   The interval between the 2007 and 2013 Gemini North and South
near-diffraction limited observations is 2,158 days.

Images reach near-diffraction limit of an 8-meter telescope and
provide sharpest images yet of the spectacular OMC1 outflow.
Figure 1 show a color rendition of the \Feii, \hh, and K$_s$
data.

Describe the color images and present obvious features.
[Fe II] dominates finger tips while H2 dominates wakes. 

There are at least 100 distinct wakes visible in the 2.12 \um\  \hh\ line, and dozens
of wake-tips (fingertips) visible mostly in the 1.64 \um\ \Feii\ line.   The \hh\  wakes exhibit
nearly parallel walls, large proper motions along their axes, and only a small amount
of spreading.    In the northwestern part of the flow, the \hh\ wakes range in diameter  from
2 to 8\arcsec\  ($7 \times 10^{14}$ to $3 \times 10^{15}$ cm). with limb brightened rims 
less than 1\arcsec ($< 3 \times 10^{14}$ cm) wide.   The approximately dozen major 
fingers in the northwest are up to 60\arcsec\   ($\sim$ 0.1 pc) long.  
The wakes in the inner part of the flow are narrower, tend to be shorter, and are 
more numerous resulting is a high degree of overlap along the line-of-sight.   
The \hh\ emission tends to disappear near the wake-tips where it is 
replaced by \Feii\ emission.    Several compact, sub-arcsecond diameter \hh\ knots are 
visible in the interiors of  wakes.      The \Feii\ images show trace the fingertips, many of which
show evidence for fragmentation.   

\subsection{Proper Motions of Selected Knots}

Comparison of sub-fields with 2007, 2008, 2009 Gemini North NIRI images show large
proper motions of  \Feii -dominated fingertips and slowly expanding  \hh 
wakes (the  \hh\ fingers).  However, there are several compact, fast moving \hh\  
knots which have sub-arc-second to arc-second diameters. 

The most prominent high-velocity compact clump (HVCC - Figure 2) exhibits a proper 
motion of $350 \pm 20$ \kms\ and over the 6 year ($1.85 \times 10^8$ sec.)  interval
between the Gemini North and South images in 2007 and 2013.  This HVCC 
is located at projected  distance of 100\arcsec\ from source I and 97\arcsec\ from 
the suspected location of the dynamic ejection of source I and BN about 500 years ago.
Assuming no deceleration, the dynamic age of this knot is 
$t_{dyn} = d / V \approx  350 \pm 30$ years.   

This HVCC  has undergone considerable  transverse spreading over the past 6 years 
indicating that it may be experiencing photometric variations or  significant deceleration.    
Several less-prominent HVCCs exhibit motions of 200 - 300 \kms . 
Measured motions of a sample (subset) of fingertips, \hh\  wakes, and compact, fast
knots are given  in Table 1.    

While the \hh\ fingers show fast $>$ 100 km/s) forward motion, 
they spread at less than  50 \kms\  orthogonal to their orientation, consistent with their
large length-to-width ratios.   The proper motions of the northwestern fingertips as well as
those located west of the OMC1 core range from 200 to 400 \kms\ with a pattern of
increasing velocity with increasing distance from OMC1.   They  have dynamic
ages consistent with ejection between 450 and 600 years ago.


\subsection{Constraints on the ejection mechanism}

Assuming the ejection occurred $\sim500$ years ago at approximately the
location of the BN/KL infrared nebula, the most distant knots to the northwest provide 
constraints on the ejecta properties.  These knots have traveled
$\sim0.28$ pc in projection in $\sim500$ years.  Thus,  the time-averaged 
velocities must be greater than   $\sim550$ \kms.   For and explosive origin, 
the fingertips proper motions should decrease linearly with decreasing distance 
from the launch region.   The northern and northwestern \Feii\ fingertips show motions
of $\sim 400$ \kms  .   While a few \hh\ features have proper motions between 300
and 350 \kms , most show lower velocities, especially closer to the OMC1 core.   
Because the faster motions are at least 20\% slower than expected for a 500 
year-old explosion,   the ejecta may have suffered some deceleration.   

Several dozen \Feii\ knots  and HVCCs seen in  \hh\  have diameters
of about  1 \arcsec\ ($\sim400$ AU) or less and are located more than 
100\arcsec\ from their ejection site.    Such clumps could have been powered by a 
faster wind that had experienced instabilities causing it to break-up into a multitude of
finger-shaped protrusions \cite{McCaughrean_MacLow1997}.    Rayleigh-Taylor
instabilities can produce fingers of fast ejecta surrounded by slower clumps of dense 
gas if the wind velocity increases with time on a time scale shorter than the mean
crossing time,  or if the wind runs into a stationary medium with a density profile 
which decreases faster than $r^{-2}$.   However, neither an accelerating wind, nor
a wind that runs down a steep density gradient would naturally explain the
approximately Hubble-flow type behavior with $V_{max}$ increasing linearly with
increasing projected distance from the source that is exhibited by the fasted ejecta 
\citep{Bally2011}.   Nor do wind models easily form compact clumps near the fingertips.

Alternatively, the wakes and fingers may be driven by compact,  high-density ejecta
(bullets or HVCCs) originating within a few AU of the massive stars in OMC1.   
In this  model, momentum conservation sets a minimum constraint
on the ejecta density because they have move through the dense gas in the Orion
molecular cloud.     Momentum conservation indicates that such knots must be denser 
than the environment into which they are moving.  A spherical clump 170 AU
in diameter that moved more than 0.2 pc into a medium with density 
$n(H_2) = 10^4$ \percc\ must have an \hh\  density  $>  10^6$ \percc\ in order
to preserves at least half of its initial ejection velocity.  If the clump is expanding
with an internal sound speed $c_s$, the Mach angle is given my $M \sim c_s/ 2 V$.
Using the observed sized of compact \Feii\ and \hh\ knots and a distance of
greater than 0.2 pc from the ejection site implies $c_s < 0.6$ \kms .   A 170 AU diameter
sphere with density $>  10^6$ \percc\ has a mass $> 10^{29}$ grams. 

The  properties of the \hh\  wakes provide constrains on the density of the ambient 
medium into which the suspected HVCCs are moving.   The wakes have widths that are 
an order-of-magnitude wider than the HVCCs and \Feii\  fingertips (2\arcsec\ to 10\arcsec\ 
with most being near the lower-end of this range).   The formation of such wide wakes requires 
that the post-shock layer forming between the forward shocks and a high-velocity,  dense 
clump  (or working surface of a jet) have a cooling length larger than the clump diameter or 
jet \citep{Blondin1990}.      The post-shock temperature immediately behind a shock is 
given by $T_{ps} = 3 \mu V_s^2/ 16 k$ where $V_s \sim $ 300 to 500 \kms\  is the forward 
shock speed.  The cooling distance is then given by $d_{cool} = V_s t_{cool} / 4$
$= 9 \mu V_s^3 / 64 n_0 \Lambda (T_{ps})$ where $\mu$ is the mean molecular weight of
the pre-shocked gas particles and $\Lambda (T_{ps})$ is the cooling function
\citep{Blondin1990}.    Numerical studies of the cooling function give
$d_{cool} \approx 4.5 \times 10^{16} V_{100}^{4.0} / n_0$  where $V_{100}$ is the
shock velocity in units of 100 \kms\ and $n_0$ is the pre-shock particle density.   
Modern numerical calculations give 
$d_{cool} \approx 5.5 \times 10^{17} V_{100}^{4.4} / n_0$  for $80$ \kms\
$< V_s <$ 1,200 \kms\ \citep{Draine2011}.  Thus, a $V_s$ = 300 \kms\ shock
moving into a density $n_0 = 10^4$ \cmq\ has a cooling length 
$L_{cool} \sim 7 \times 10^{15}$ cm (470 AU).  Thus, for densities between $10^3$ and
$10^4$, cooling lengths will correspond to 1.1  to 11\arcsec\ in the frame of the shock 
at the distance of Orion, larger than the HVCCs and \Feii\  knots.

For a very dense HVCCs, the hot ($\sim$ 6 MK) plasma will squirt sideways to produce a wide
bow-shaped wake.   The sideway expansion into the surrounding medium drives a slower
($V_{side} < $ 50 \kms ) shock where the observed \hh\ emission in the wakes is produced.
In the OMC1 rest frame the 30 to 300 year cooling time corresponds to a wake-length 
$L_{wake} \sim 3 \times 10^{16}$ to $3 \times 10^{17}$ cm, comparable to the lengths of
fingers in the north and west parts of the OMC1 outflow.

While most fingertips are invisible \citet{Doi2002}  measured visual
wavelength proper motions for those which protrude into the photon-dominated region
behind the Orion Nebula.  
\citet{Grosso2006} detected X-rays from the wake of  HH~210, the highest proper motion
finger ($\sim$ 425 \kms ) in the OMC1 outflow and one of the relatively few visible at visual 
wavelengths, thereby demonstrating that at least some of the fingers contain hot,  X-ray 
emitting plasma.

Predictions for ALMA (dust and gas column density)

Comparison of selected fingers with models from ENZO code 
(Devin Silvia) 

Is there any correlation between the complex structure of OMC1 revealed by
SCUBA?  First impression is no .. but there is mechanical `shadowing' by the
OMC1 ridge: fingers towards NE and SW are shorter; more resistance or more
recent ejecta from the young SiO / H2O flow from source I?

\subsection{Other YSOs in the field}

The [FeII] bipolar jet, the silhouette disk.
These YSOs (possibly other embedded ones), may be 
impacted by phase ejecta and shocks! 

\section{Conclusions}

Summary of results:

\begin{itemize}

\item Compact high-density knots may be located at
finger tips (from propagation constraint).

\item Some have reverse shocks that light-them up in H2 despite
V > 200 km/s => independent constraint on density.

\item Predict that ALMA will see CO wakes, and detect 
compact knots of hot gas (SiO? HCO+?, high-J CO?)
from "bullets"
\end{itemize}



\bibliographystyle{apj_w_etal}
\bibliography{Orion}

\input{Orion_figures}

\input{Orion_table1}


\end{document}
