
%%%%%%%%%%%%%%%%%%%%
%                                                                  % 
%      Orion GSAOI                                    %
%                                                                 %
% Version controlled at :
% https://github.com/keflavich/OrionOutflowGEMS
%%%%%%%%%%%%%%%%%%%%

\documentclass[12pt,preprint]{aastex}
%\documentclass{$HOME/A_WORK_DIRECTORIES/Latex/emulateapj}
%$
%% manuscript produces a one-column, single-spaced document:

\usepackage{rotating}

\citestyle{aa}

%$

%\documentclass[preprint2]{aastex}
%% preprint2 produces a double-column, single-spaced document:

%\documentclass[manuscript]{aastex}
%% preprint2 produces a one-column, double-spaced document:

\newcommand{\vdag}{(v)^\dagger}
\newcommand{\myemail}{skywalker@galaxy.far.far.away}
\newcommand{\lsim}{{_{<}\atop^{\sim}}}
\newcommand{\gsim}{{_{>}\atop^{\sim}}}
\newcommand{\etal}{{et al.\/}}
\newcommand{\ie}{{\em ie.\/}}
\newcommand{\cmq}{cm{$^{-3}$}}
\newcommand{\per}{$^{\rm{-1}}$}
\newcommand{\tc}{{$\theta^1$~Orionis~C}}
\newcommand{\msol}{M{$_{\odot}$}}
\newcommand{\lsol}{L{$_{\odot}$}}
\newcommand{\kms}{km~s{$^{-1}$}}
\newcommand{\hii}{H~{\sc ii}}
\newcommand{\Hii}{H~{\sc ii}}
\newcommand{\Ha}{\mbox{H$\alpha$}}
\newcommand{\sii}{S~{\sc ii}}
\newcommand{\Feii}{[Fe~{\sc ii}]}
\newcommand{\oi}{[O~{\sc i}]}
\newcommand{\nii}{[N~{\sc ii}]}
\newcommand{\oiii}{[O~{\sc iii}]}
\newcommand{\mgii}{[Mg~{\sc ii}]}
\newcommand{\tco}{{$^{13}$CO}}
\newcommand{\CO}{{$^{12}$CO}}
\newcommand{\Tco}{{$^{12}$CO}}
\newcommand{\co}{C{$^{18}$O}}
\newcommand{\Lsol}{L$_{\odot}$}
\newcommand{\Vlsr}{V$_{LSR}$}
\newcommand{\Msol}{M$_{\odot}$}
\newcommand{\um}{$\mu$m}
%\newcommand{\C18o}{C$^{18}$O($1\rightarrow 0$)}
\newcommand{\Check}{{\bf ???}}
\newcommand{\mum}{\ensuremath{\mu \mathrm{m}}}
\newcommand{\flux}{flux density}
\newcommand{\solar}{\ensuremath{\odot}}

\newcommand{\msun}{\ensuremath{M_{\odot}}}	 	%  Msun
\newcommand{\lsun}{\ensuremath{L_{\odot}}}			%  Lsun
\newcommand{\lbol}{\ensuremath{L_{\mathrm{bol}}}}	%  Lbol
\newcommand{\ks}{K\ensuremath{_{\mathrm{s}}}}		%  Ks
\newcommand{\hh}{\ensuremath{\textrm{H}_{2}}}			%  H2
\newcommand{\water}{H$_{2}$O}		%  H2O
\newcommand{\feii}{\ion{Fe}{2}}		%  FeII
\newcommand{\sqcm}{cm$^{2}$}		%  cm^2
\newcommand{\percc}{\ensuremath{\textrm{cm}^{-3}}}
\newcommand{\persc}{\ensuremath{\textrm{cm}^{-2}}}
\newcommand{\persr}{\ensuremath{\textrm{sr}^{-1}}}
\newcommand{\peryr}{\ensuremath{\textrm{yr}^{-1}}}
\newcommand{\persec}{\ensuremath{\textrm{s}^{-1}}}
\newcommand{\htwo}{\ensuremath{\textrm{H}_2}}    % H2
\newcommand{\Htwo}{\ensuremath{\textrm{H}_2}}    % H2
\newcommand{\HtwoO}{\ensuremath{\textrm{H}_2\textrm{O}}}   
\newcommand{\htwoo}{\ensuremath{\textrm{H}_2\textrm{O}}}   
\newcommand{\ha}{\ensuremath{\textrm{H}\alpha}}
\newcommand{\hb}{\ensuremath{\textrm{H}\beta}}
\newcommand{\twelveco}{\ensuremath{^{12}\textrm{CO}}}
\newcommand{\thirteenco}{\ensuremath{^{13}\textrm{CO}}}
\newcommand{\ceighteeno}{\ensuremath{\textrm{C}^{18}\textrm{O}}}

	% End definitions



\slugcomment{DRAFT: \today}
\shorttitle{Orion GEMS GSAOI}
\shortauthors{Bally et al.}


\begin{document}

\title{The Orion Fingers:   Near-IR Adaptive Optics Imaging of
an Explosive Protostellar Outflow}

\author{
               John Bally\altaffilmark{1},               
               Adam Ginsburg\altaffilmark{2},
	    }
	    
\affil{{$^1$}{\it{
      Department of Astrophysical and Planetary Sciences,\\
      University of Colorado, UCB 389, \\
      Boulder, CO 80309}      
      \email{John.Bally@colorado.edu}}  }
\affil{{$^2$}{\it{
ESO Headquarters
Karl-Schwarzschild-Str. 2
85748 Garching bei M�nchen
Germany
}      
      \email{Adam.Ginsburg@eso.org}}  }


\keywords{ISM: - molecular clouds --
ISM: - shocks, outflows
ISM: individual -- Orion Nebula, OMC1
stars: formation -- }

\begin{abstract}
We present new narrow-band \hh ,   [\feii], and broad-band K$_s$ images of the 
Orion  OMC1 outflow  obtained with the Gemini South multi-conjugate 
adaptive optics (AO) system and near-infrared imager GSAOI.   These  images 
reach the diffraction limit of the 8-meter telescope at 2 \mum\ of about 0.05\arcsec  .
Comparison with previous AO-assisted observations of sub-fields with the Gemini
North telescope between 2007 to 2009 and ground-based observations going back
to 1999 enable measurements of proper motions of \hh\ and \Feii\ features with 
unprecedented precision in portions of the outflow.    Several sub-arcsecond \hh\ 
features and many \Feii\   `fingertips' on the projected outskirts of the flow show 
proper motions of $\sim$300 \kms.    The propagation of compact  knots (`bullets') 
as far as 140\arcsec\ from their ejection  sites through the dense OMC1 core sets 
a lower bound on their densities.  The density, mass, and energy constraints are  
consistent with the  disruption  of dense  circumstellar disks within a few AU
of massive stars during a final multi-body dynamic encounter that ejected  the BN 
object and radio source I from the OMC1 about 500 years ago.  

Orion is frequently, and appropriately, used as a template for understanding
other massive star forming regions.  These new data provide excellent evidence
for shock-heated, megaKelvin gas generating \hh\ emission in interaction
regions with a molecular cloud.  Such interaction zones are likely to be a
common feature in high-mass-star forming regions and will likely dominated the
\hh\ emission from such regions.
\end{abstract}



\section{Introduction} 

The BN/KL region behind the Orion Nebula, located at a distance of about
414 pc \citep{Menten2007} contains a spectacular, 
wide opening-angle, arcminute-scale  outflow emerging from the
OMC1 cloud core.  The flow is traced by the millimeter and
sub-millimeter emission lines of molecules 
such as CO, CS, SO, SO$_2$, and HCN  that exhibit  broad  
($>$ 100 km s$^{-1}$)  emission line wings 
\citep{KwanScoville1976, WisemanHo1996, FuruyaShinnaga2009},  
high-velocity OH, H$_2$O,  and SiO maser  emission \citep{Genzel1981,Greenhill1998}, 
and bright shock-excited `fingers' of  \hh\   and `fingertips' of 1.64 \um\  \Feii\   emission
\citep{AllenBurton93, Bally2011}.    The OMC1 outflow with a southeast 
(red-shifted) - northwest (blue-shifted) axis
contains  at least 8 M$_{\odot}$ of  accelerated gas with a median velocity of about 
20 km s$^{-1}$.     Interferometric CO images, H$_2$O and the 18 km~s$^{-1}$ 
SiO masers,  and dense-gas tracers such as  thermal SiO reveal a 
smaller (8\arcsec\  long) and younger ($\sim$ 200 
year old) outflow along a  northeast-southwest axis emerging from radio source I  
orthogonal to the arc-minute-scale  CO outflow  \citep{BeutherNissen2008,Plambeck2009}.    
The  momentum and kinetic   energy content of these flows is at least  
$160$ M$_{\odot}$ km s$^{-1}$ and  $4 \times 10^{46}$  ergs  \citep{Snell1984} to  
$4 \times 10^{47}$ ergs  \citep{KwanScoville1976}.     \citet{Zapata2009}  presented a
CO J = 2--1 interferometric study and found a dynamic age of  about $500$ 
years for the larger OMC1 outflow.   They noted its impulsive nature, that its  structure is 
different from accretion-disk  powered  flows,  and that it  originated several arc-seconds 
north of the  OMC1 hot-core.   

The OMC1 outflow  contains  a complex of shocks which may indicated an explosive
origin.     High precision astrometric measurements have shown that
the three brightest radio-emitting stars in OMC1, sources BN, I, and possibly
source n, have  proper motions  of 25, 15, and 26 km s$^{-1}$ 
away from a region less than 500 AU in diameter from which they were ejected 
about  500 years ago (Rodriguez et al. 2005; Gomez et al. 2005; 2008).   
\citet{Bally2011} and \citet{Goddi2011} proposed that the explosion was triggered 
by the dynamical rearrangement of a nonhierarchical system of massive stars in 
OMC1 which resulted in the formation of a compact, massive binary (most likely 
source I) and the ejection of both BN,   source I, and possibly source n.  
Proper motion measurements show that the fastest components in the 
OMC1 fingers have a dynamic age of about 500 years \citep{Bally2011}, consistent 
with this scenario.   

On the other hand, \citet{Tan2004} proposed that the decay 
occurred about 4,000 years ago in the Trapezium cluster located in the center of
the Orion Nebula and that the OMC1 explosion was
triggered by the close passage of the BN object through the OMC1 core.   Although 
this scenario requires a highly unlikely close encounter of  BN with source I, 
\citet{Tan2012} show that the the parameters of the Trapezium and BN are compatible with 
this scenario.
 
Though rare, the explosive outflow morphology of the  OMC1 outflow is not unique;
other likely examples \citep{Bally2011}.   However, Orion BN/KL is the nearest and least 
obscured,and thus most accessible for high-resolution studies.  
Here, we present 0.06 to 0.1 arc second resolution images of the 
entire OMC1 outflow complex in the 1.64 \um\ \Feii\ and 2.12 \um\ \hh\  narrowband filters 
and a broad-band K$_s$ obtained with adaptive optics (AO) on the Gemini South  8 
meter telescope.    This data is combined with  older AO-assisted observations
obtained on Gemini North and natural seeing-limited images acquired with a variety
of other telescope to measure new proper  motions. 


\section{Observations}

\subsection{Gemini South GEMS}

The Gemini Multi-conjugate adaptive optics System (GeMS) at the Gemini South telescope
in Cerro Pachon is the first and only sodium-based multi-Laser Guide Star (LGS) adaptive
optics system \citep{Rigaut2014,Rigaut2012,Neichel2014,Neichel2013,dOrgeville2012}.
GeMS works with a LGS constellation of 5-spots:  4 of the LGS spots are at the corners of a
60\arcsec\ square, with the 5-th positioned in the center.  The Adaptive Optics (AO)  bench
called Canopus is mounted on one of the f/16 Cassegrain ports.    Gemini South 
Adaptive Optics Imager (GSAOI) is a �wide-field� (85\arcsec\ by 85\arcsec\ field of view) 
camera designed to work at the diffraction limit of the 8-meter telescope in the near-infrared.
Three 85\arcsec\  fields were observed in OMC1 between 30 December 2012 and
28 February 2013 using GSAOI.  Observations of each field were obtained though 1\%
bandpass narrow-band filters centered on the 1.644 $\mu$m [FeII] and 2.122 $\mu$m
H$_2$ emission lines and the broad-band K$_s$ filter.  
The corrected images have FWHM diameters of 0.06\arcsec\ to  0.1\arcsec , providing the 
highest angular resolution images of the BN/KL outflow ever obtained in the near-IR. 

Each field was imaged in each filter with a 5 point dither pattern to fill-in gaps between the 
four 2048 by 2048 pixel arrays in GSAOI.    In the two narrow band filters, exposure times of 
30 seconds per image were used; exposure times were 10 seconds 
per frame were used in the broad-band filter.

%xxxxxx  Describe NUMBER OF EXPOSURES, and EXPTIME.  Describe reduction steps and
%difficulties.

Exposure times were 43.4 seconds per exposure for \hh, 43.0 for [Fe II], and
15.0 for Ks.  10 exposures were taken in each filter for a total of 430s
on-source in the narrow-band filters and 150s in the continuum filter.


Data were processed with the Gemini pipeline.  However, additional astrometric
corrections were required.   Individual exposures were first registered to the
\citet{Muench2002a} catalog sources to acquire a world coordinate system with
RMS pointing error $\sim0.1$ \arcsec .     A new catalog of relative star positions
was generated from a preliminary aligned and co-added stack of images and used
to derive a distortion map for GSAOI.    The individual distortion corrected images were 
re-aligned and co-added to form the final mosaic in each filter.  

\subsection{Gemini North Altair}

The Gemini North 8 meter telescope was used to observe the OMC1 region using the
Altair AO system with the NIRI near-IR camera through 1\% narrow-band \Feii , \hh , and
broad-band K$_s$ filters.    NIRI was used in a configuration which
delivers a  40\arcsec\ field of view.  In the narrow-band filters, a dithered set of five to 10 
30 second duration exposures were obtained.    A similar set of 10 second exposures were
acquired in the K$_s$ filter.     In 2007, only the `\hh\  fingers' region was observed as part
of the commissioning of the Altair AO system using the NIRI camera.   During 2008 and 2009,
we intended to image a 3 $\times$ 4 point grid to cover the full extent of the BN/KL outflow.
However, on 5 and 8 fields were actually observed in 2008 and 2009, respectively.    Only
the  `\hh\  fingers'  field was observed during each of the three years.   A summary of the 
observations is given in Table~1.   The angular resolution of the NIRI images ranges from 0.1 
to 0.2\arcsec . 

AO images  were obtained with Gemini North on MJD 54165, MJD 54753, and MJD 55138 
between 2007 and 2009,  and Gemini South on MJD 56323. 
In the analysis presented here, the NIRI images were registered to the final GSAOI mosaic
using IRAF tasks GEOMAP and GEOTRAN applied to unsaturated field stars.    
Proper motions were determined by marking  the photocenters of features on the multi-epoch
images.     Images of the OMC1 outflow obtained with the 
Subaru 8 meter telescope on MJD = 51484 in 1999 \citep{Kaifu2000},
the Apache Point Observatory 3.5 meter on MJD = 53331 in 2004  \citep{Bally2011}
are also used for analysis.    These observations span an interval of 4,839 days.   

\section{Results}

The 2013 epoch GSAOI images presented here reach the near-IR diffraction limit of an 
8-meter telescope and provide sharpest views obtained thus far of the 
entire  OMC1 BN/KL  outflow.      The image (Figure \ref{fig1}) shows dozens of \hh\ fingers 
tipped with \Feii\  emission extending from about 30\arcsec\ to 140\arcsec\ from the OMC1 core.  
For the analysis of dynamic ages for various features, we assume that all features originated
from J2000 = 05:35:14.350, -05:22:28.50,  the suspected location from which the BN object 
and radio source I  were ejected about 500 years ago  \citep{Gomez2008}.  
The two brightest \Feii\  bow shocks  correspond to the Herbig-Haro objects 
HH~201  and HH~210 located 60\arcsec\ northwest and  113\arcsec\  north of  OMC1
\citep{Gull1973,MunchTaylor1974,Canto1980,AxonTaylor1984}.   
These shocks are visible on ground-based and Hubble Space Telescope images in
\oi  ,  H$\alpha$,  \nii\, and \sii .  However, they  only exhibit very faint  \hh\ emission
\citep{Graham2003}, indicating that
they lie in the mostly atomic photon-dominated region (PDR) located between the Orion
Nebula's ionization front and the background molecular cloud.    
The most prominent  \hh\ finger consist of  multiple \Feii\  finger-tips trailed by \hh\ wakes,  
has an orientation of PA $\sim$ 340 to 345 \arcdeg  ,  and can be traced from
about 50\arcsec\  to about 135\arcsec\ from the ejection center.  This wake consists of  
at least a dozen  nested \hh\ bow shocks tipped  with \Feii\ emission regions.      
The brighter  \Feii\  knots  corresponding to HH~205 through 209
are associated with the tips of  a train of \hh\  wakes propagating toward position angle
(PA) $\sim$  340 to 350\arcdeg .   These HH objects are associated with the 
PA $\sim$ 340 to 345 \arcdeg\  finger.

More than 120 distinct wakes are visible in the 2.12 \um\  \hh\  images presented here. 
The \hh\  wakes exhibit nearly parallel walls, large proper motions along their axes
\citep{Bally2011}.      
Figure \ref{fig2}  shows a median filtered version of the 2013 GSAOI mosaic.   
This high-pass filtered image was  created by convolving the original images with a 51 pixel kernel (1\arcsec ; each pixel is   0.02\arcsec\  on a side) using IRAF function MEDIAN, and subtracting 
the the result from the original image.   Vectors were drawn from the suspected ejection site of 
radio sources BN and source I (the coordinates are given  above) to each \hh\ or \Feii\ fingertip.    
The \hh emission becomes too contused within $\sim$30\arcsec\  of the suspected 
ejection location due to the complex of multiple overlapping features.    
The natural seeing limited 1999 epoch Subaru image
from \citet{Kaifu2000} and the 2005 epoch image from \citet{Bally2011} were used to 
trace additional \hh\ fingers beyond the boundaries of the GSAOI image.    The dashed vector near 
the top marks a chain of  \hh\ knots and bow shocks with proper motions nearly orthogonal to the 
northern fingers suspected to trace a background flow originating east of the present field.
This flow is also seen faintly in visual wavelength images and exhibits large proper motions.

While the fingertips tend to be faint
or invisible in \hh , most of the the \Feii\ emission in the BN/KL outflow originates from
the fingertips.   Dozens of  wake-tips (fingertips) are visible  in the 1.64 \um\ \Feii\ line.   
The two cyan vectors point to the two  brightest \Feii\ features, HH~201 and 210.   
These fingertips are very faint in \hh .   Figure \ref{fig4} shows a color version of 
the `\hh\ fingers'  region in 2013 from GSAOI.

In the northwestern  part of the flow, the \hh\ wakes range in diameter  from 2 to 8\arcsec\  
($7 \times 10^{14}$ to $3 \times 10^{15}$ cm)  with limb brightened rims 
less than 1\arcsec ($< 3 \times 10^{14}$ cm) wide (Figure \ref{fig3}).   
The half dozen major  finger clusters in the northwest are up to 60\arcsec\   ($\sim$ 0.1 pc) long.  
The wakes in the inner part of the flow are narrower, tend to be shorter, and are 
more numerous resulting is a high degree of overlap along the line-of-sight.   
The \hh\ emission tends to be fainter or disappears near the wake-tips where it is 
replaced by \Feii\ emission.     \Feii\  emission is only found near the fingertips, 
some of which show evidence for fragmentation. 

\subsection{Proper Motions of Selected Knots}

Previous analyses of  multi-epoch ground-based images have shown that the overall expansion 
pattern of the BN/KL outflow is a `Hubble flow' with the proper motions being approximately 
proportional to the projected distance from OMC1 \citep{JonesWalker1985,LeeBurton2000,Bally2011}.    

\citep{Kaifu2000}

Figure \ref{fig3} shows a difference

Comparison of sub-fields with 2007, 2008, 2009 Gemini North NIRI images show large
proper motions of  \Feii -dominated fingertips and slowly expanding  \hh 
wakes (the  \hh\ fingers).  However, there are several compact, fast moving \hh\  
knots which have sub-arc-second to arc-second diameters. 
The most reliable proper motions were measured in the `\hh\ fingers' field where we have
AO-assisted images from both 2013 and 2007 separated by 2,125 days.  
Figure \ref{fig4} shows a color composite image of this field.  
and \ref{fig6} show the differences between the  \hh\  and \Feii\  images taken in 3013 and 2007.

A difference image formed by subtracting a de-distorted, intensity matched, and registered
2013 GSAOI image from the 2007 NIRI image (Figures \ref{fig5})
shows that the \hh\ wakes are spreading with velocities ranging from 20 to 80 \kms .




The most prominent high-velocity compact clump (HVCC - Figure 2) exhibits a proper 
motion of $350 \pm 20$ \kms\ and over the 6 year ($1.85 \times 10^8$ sec.)  interval
between the Gemini North and South images in 2007 and 2013.  This HVCC 
is located at projected  distance of 100\arcsec\ from source I and 97\arcsec\ from 
the suspected location of the dynamic ejection of source I and BN about 500 years ago.
Assuming no deceleration, the dynamic age of this knot is 
$t_{dyn} = d / V \approx  350 \pm 30$ years.   

This HVCC  has undergone considerable  transverse spreading over the past 6 years 
indicating that it may be experiencing photometric variations or  significant deceleration.    
Several less-prominent HVCCs exhibit motions of 200 - 300 \kms . 
Measured motions of a sample (subset) of fingertips, \hh\  wakes, and compact, fast
knots are given  in Table 1.    

While the \hh\ fingers show fast $>$ 100 km/s) forward motion, 
they spread at less than  50 \kms\  orthogonal to their orientation, consistent with their
large length-to-width ratios.   The proper motions of the northwestern fingertips as well as
those located west of the OMC1 core range from 200 to 400 \kms\ with a pattern of
increasing velocity with increasing distance from OMC1.   They  have dynamic
ages consistent with ejection between 450 and 600 years ago.

Comparison of HST images in [OI] and [SII] taken on MJD 50170 show that the HVCC
is located a few arc second south of HH~207. 

HH 205, 206, and 207 trace the visual counterpart of the PA $\sim$ 340 to 345\arcdeg\ finger.
The \Feii\ counterpart of HH~207 exhibits a prompter motion of 214 \kms . 

The side jet isa visual HH object with large motions.  

d132-042 (Smith et al. 2005) has an \Feii\ micro-jet.    

\subsection{Constraints on the ejection mechanism}

Assuming the ejection occurred $\sim500$ years ago at approximately the
location of the BN/KL infrared nebula, the most distant knots to the northwest provide 
constraints on the ejecta properties.  These knots have traveled
$\sim0.28$ pc in projection in $\sim500$ years.  Thus,  the time-averaged 
velocities must be greater than   $\sim550$ \kms.   For and explosive origin, 
the fingertips proper motions should decrease linearly with decreasing distance 
from the launch region.   The northern and northwestern \Feii\ fingertips show motions
of $\sim 400$ \kms  .   While a few \hh\ features have proper motions between 300
and 350 \kms , most show lower velocities, especially closer to the OMC1 core.   
Because the faster motions are at least 20\% slower than expected for a 500 
year-old explosion,   the ejecta may have suffered some deceleration.   

Several dozen \Feii\ knots  and HVCCs seen in  \hh\  have diameters
of about  1 \arcsec\ ($\sim400$ AU) or less and are located more than 
100\arcsec\ from their ejection site.    Such clumps could have been powered by a 
faster wind that had experienced instabilities causing it to break-up into a multitude of
finger-shaped protrusions \cite{McCaughrean_MacLow1997}.    Rayleigh-Taylor
instabilities can produce fingers of fast ejecta surrounded by slower clumps of dense 
gas if the wind velocity increases with time on a time scale shorter than the mean
crossing time,  or if the wind runs into a stationary medium with a density profile 
which decreases faster than $r^{-2}$.   However, neither an accelerating wind, nor
a wind that runs down a steep density gradient would naturally explain the
approximately Hubble-flow type behavior with $V_{max}$ increasing linearly with
increasing projected distance from the source that is exhibited by the fasted ejecta 
\citep{Bally2011}.   Nor do wind models easily form compact clumps near the fingertips.

Alternatively, the wakes and fingers may be driven by compact,  high-density ejecta
(bullets or HVCCs) originating within a few AU of the massive stars in OMC1.   
In this  model, momentum conservation sets a minimum constraint
on the ejecta density because they have move through the dense gas in the Orion
molecular cloud.     Momentum conservation indicates that such knots must be denser 
than the environment into which they are moving.  A spherical clump 170 AU
in diameter that moved more than 0.2 pc into a medium with density 
$n(H_2) = 10^4$ \percc\ must have an \hh\  density  $>  10^6$ \percc\ in order
to preserves at least half of its initial ejection velocity.  If the clump is expanding
with an internal sound speed $c_s$, the Mach angle is given my $M \sim c_s/ 2 V$.
Using the observed sized of compact \Feii\ and \hh\ knots and a distance of
greater than 0.2 pc from the ejection site implies $c_s < 0.6$ \kms .   A 170 AU diameter
sphere with density $>  10^6$ \percc\ has a mass $> 10^{29}$ grams. 

The  properties of the \hh\  wakes provide constrains on the density of the ambient 
medium into which the suspected HVCCs are moving.   The wakes have widths that are 
an order-of-magnitude wider than the HVCCs and \Feii\  fingertips (2\arcsec\ to 10\arcsec\ 
with most being near the lower-end of this range).   The formation of such wide wakes requires 
that the post-shock layer forming between the forward shocks and a high-velocity,  dense 
clump  (or working surface of a jet) have a cooling length larger than the clump diameter or 
jet \citep{Blondin1990}.      The post-shock temperature immediately behind a shock is 
given by $T_{ps} = 3 \mu V_s^2/ 16 k$ where $V_s \sim $ 300 to 500 \kms\  is the forward 
shock speed.  The cooling distance is then given by $d_{cool} = V_s t_{cool} / 4$
$= 9 \mu V_s^3 / 64 n_0 \Lambda (T_{ps})$ where $\mu$ is the mean molecular weight of
the pre-shocked gas particles and $\Lambda (T_{ps})$ is the cooling function
\citep{Blondin1990}.    Numerical studies of the cooling function give
$d_{cool} \approx 4.5 \times 10^{16} V_{100}^{4.0} / n_0$  where $V_{100}$ is the
shock velocity in units of 100 \kms\ and $n_0$ is the pre-shock particle density.   
Modern numerical calculations give 
$d_{cool} \approx 5.5 \times 10^{17} V_{100}^{4.4} / n_0$  for $80$ \kms\
$< V_s <$ 1,200 \kms\ \citep{Draine2011}.  Thus, a $V_s$ = 300 \kms\ shock
moving into a density $n_0 = 10^4$ \cmq\ has a cooling length 
$L_{cool} \sim 7 \times 10^{15}$ cm (470 AU).  Thus, for densities between $10^3$ and
$10^4$, cooling lengths will correspond to 1.1  to 11\arcsec\ in the frame of the shock 
at the distance of Orion, larger than the HVCCs and \Feii\  knots.

For a very dense HVCCs, the hot ($\sim$ 6 MK) plasma will squirt sideways to produce a wide
bow-shaped wake.   The sideway expansion into the surrounding medium drives a slower
($V_{side} < $ 50 \kms ) shock where the observed \hh\ emission in the wakes is produced.
In the OMC1 rest frame the 30 to 300 year cooling time corresponds to a wake-length 
$L_{wake} \sim 3 \times 10^{16}$ to $3 \times 10^{17}$ cm, comparable to the lengths of
fingers in the north and west parts of the OMC1 outflow.

While most fingertips are invisible \citet{Doi2002}  measured visual
wavelength proper motions for Herbig-Haro (HH) object which protrude into the 
photon-dominated region behind the Orion Nebula.    Two of the brightest \Feii\ features
correspond to HH~201 northwest of OMC1 and 210 north of OMC1.    citet{Doi2002}
meadured proper motions of 312 to 315 \kms\ for various components of HH~201 and
and 309 to 425 \kms\ for knots in HH~210.     \citet{Grosso2006} detected X-rays from the 
wake of  HH~210, which is the highest proper motion finger  in the OMC1 outflow and one 
of the relatively few visible at visual 
wavelengths, thereby demonstrating that at least some of the fingers contain hot,  X-ray 
emitting plasma.   Additional HH objects are associated with the Orion fingers including
HH 205 to 209, and HH 601 to 607  with proper motion velocities ranging 
from  100 to 300 \kms\  \citep{Doi2002}.  All of these features have large negative radial velocties.
For example, HH~201 has \Vlsr\ $\sim$ $-$260 to $-$284 \kms  \citep{Doi2004}.

Predictions for ALMA (dust and gas column density)
% Maybe we can leave this out now; ALMA data will be coming so soon...

Comparison of selected fingers with models from ENZO code 
(Devin Silvia) 

Is there any correlation between the complex structure of OMC1 revealed by
SCUBA?  First impression is no .. but there is mechanical `shadowing' by the
OMC1 ridge: fingers towards NE and SW are shorter; more resistance or more
recent ejecta from the young SiO / H2O flow from source I?

\subsection{Other YSOs in the field}

The [FeII] bipolar jet, the silhouette disk.
These YSOs (possibly other embedded ones), may be 
impacted by phase ejecta and shocks! 

V* V2270 Ori (05 35 15.3937	 -05 21 14.112) is driving a bipolar [Fe II] jet.

\section{Conclusions}

Summary of results:

\begin{itemize}

\item Compact high-density knots may be located at
finger tips (from propagation constraint).

\item Some have reverse shocks that light-them up in H2 despite
V > 200 km/s => independent constraint on density.

\item Predict that ALMA will see CO wakes, and detect 
compact knots of hot gas (SiO? HCO+?, high-J CO?)
from "bullets"
\end{itemize}



\bibliographystyle{apj_w_etal}
\bibliography{Orion}

\input{Orion_figures}
% http://i.imgur.com/FwquUQk.png

\input{Orion_table1}

\input{Orion_table2}


\end{document}
